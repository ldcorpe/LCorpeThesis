\chapter{Theory}
\label{chap:theory}

\section{Introduction}

The \SM of particle physics was developed in the early 1970s. %during the second half of the 20$^{\text{th}}$ century. 
It has been immensely successful, accurately describing all processes so far encountered in high energy physics~\cite{PDGBooklet}. 
%The \SM is a \QFT, in particular a renormalisable gauge theory. 
In the \SM, the fundamental elements of matter, as well as the forces which govern their interaction, are represented as relativistic quantum fields, the excitations of which are manifested as particles. These particles and forces are reviewed in \Sec~\ref{sec:th:particlesandforces}. The \SM places all the forces except for gravity in the same framework, and unites the electromagnetic and weak forces as the electroweak force, as described in \Sec~\ref{sec:th:gauge}. The Brout-Englert-Higgs mechanism, which explains the breaking of the underlying symmetry between these forces and leads to the prediction of an observable particle (the Higgs boson), is described in \Sec~\ref{sec:th:ewsb}. %All the particles postulated by the \SM have now been discovered, capped by the discovery of the Higgs boson in 2012.
An overview of past searches for the Higgs boson is presented in \Sec~\ref{sec:th:higgs_searches}, along with a discussion on the main Higgs boson production and decay processes at the \LHC in \mbox{\Sec\s}~\ref{sec:th:higgs_production_modes} and~\ref{sec:th:higgs_decays} respectively.

%In order to do this, a mechanism ss required to permit the $W^{\pm}$ and $Z$ vector bosons to have mass while allowing the photon to remain masslesss. Such a theory was independently proposed by several theorists~\cite{BroutEnglert,Higgs1,Higgs2,Kibble1,Higgs3,Kibble2}, and is commonly referred to as the Higgs mechanism. In 1964, Higgs postulated that one outcome of this mechanism was that it should yield an observable particle, the Higgs boson~\cite{Higgs2}.

\section{The Standard Model of particle physics}

\subsection{Particles and Forces}
\label{sec:th:particlesandforces}
 In the \SM, matter is made up of \SpinHalf particles, called \emph{fermions}. Fermions come in two types: those which interact exclusively via the electroweak force, the \emph{leptons}, and those which can also interact via the strong nuclear force, the \emph{quarks}. The fermions can be arranged into three \emph{generations}.
 %which are identical copies of each other, aside from the masses of the constituent particles. 
 %The first generation includes the electron (\Pe), the electron neutrino (\Pnue), the up quark (\Pup) and the down quark (\Pdown). The second generation includes the muon (\Pmu), the muon neutrino (\Pnum), the charm quark (\Pcharm) and the strange quark (\Pstrange). The third and final generation includes the tau (\Ptau), the tau neutrino (\Pnut), the truth or top quark (\Ptop) and the beauty or bottom quark (\Pbottom). 
%Each of the particles mentioned in the table has a corresponding \emph{antiparticle}, with the same mass and opposite quantum numbers.
The \SM fermions and their properties are displayed in \Tab~\ref{tab:th:fermions}.
Each of the particles mentioned in the table has a corresponding \emph{antiparticle}, with the same mass and opposite charge.

\begin{table}[h!]
 \resizebox{\textwidth}{!}{
\begin{tabular}{ |c | c l | c l | c l | }
\hline
type & \multicolumn{2}{|c}{ Generation I} & \multicolumn{2}{|c}{ Generation II} & \multicolumn{2}{|c|}{ Generation III} \\
\hline
  \multirow{4}{*}{leptons} & \Large{\Pe} & $m=0.511\MeV$ & \Large{\Pmu}  &   $m=105\MeV$ &   \Large{\Ptau}  &   $m=1.777\GeV$  \\
                           &    electron &   $q=-1$      &  muon         & $q=-1$        &  tau      & $q=-1$   \\ \cline{2-7}
                           & \Large{\Pnue}       & $m\sim0\MeV$ & \Large{\Pnum}  &   $m\sim0\MeV$ &   \Large{\Pnut}  &   $m\sim0\MeV$  \\
                           &  electron neutrino  &   $q=0$      & muon neutrino  & $q=0$          &  tau neutrino    & $q=0$   \\
 \hline 
  
  \multirow{4}{*}{quarks} & \Large{\Pup} & $m=2.3\MeV$         & \Large{\Pcharm}  &   $m=1.275\GeV$  &   \Large{\Ptop}  &   $m=173\GeV$  \\
                          &      up      &   $q=+\frac{2}{3}$  & charm            & $q=+\frac{2}{3}$ & top or truth    & $q=+\frac{2}{3}$   \\ \cline{2-7}
                           & \Large{\Pdown} & $m=4.8\MeV$      & \Large{\Pstrange} &   $m=95\MeV$      &   \Large{\Pbottom}  &   $m=4.18\GeV$  \\
                           &        down    & $q=-\frac{1}{3}$ & strange           & $q=-\frac{1}{3}$  &  bottom or beauty   & $q=-\frac{1}{3}$   \\
 \hline 
  \end{tabular}
}
 \caption[The fundamental particles of the SM. The mass $m$ and electric charge $q$ are indicated for each particle\quad\cite{PDGBooklet}. The uncertainties on the masses have been omitted, although some are large.]{The fundamental particles of the SM. The mass $m$ and electric charge $q$ are indicated for each particle~\cite{PDGBooklet}. The uncertainties on the masses have been omitted, although some are large.} 
\label{tab:th:fermions}
\end{table}

The matter particles interact via the fundamental forces. There are four known fundamental forces in the universe: the electromagnetic force, the weak nuclear force, the strong nuclear force and the gravitational force. The gravitational force is many orders of magnitude weaker than any of the other forces, and therefore has a negligible effect on the interactions of the \SM particles in high energy physics experiments. Furthermore, no adequate quantum theory of gravity currently exists, so it cannot be easily included in the \SM. Consequently, the \SM describes only with the strong, weak and electromagnetic forces. In the \SM, the fundamental forces, are represented by the exchange of spin-1 \emph{mediator particles}, the \emph{vector bosons}. 
%The electromagnetic force is mediated by the photon (\Pphoton). The weak nuclear force is mediated by the Z-boson (\PZ) and the W-bosons (\PWplus and \PWminus). The strong nuclear force is mediated by the gluons (\Pgluon), of which there are eight, one for each possible linearly independent combination of colour charge (the strong force equivalent of electric charge).
The forces described by the \SM are listed in \Tab~\ref{tab:th:bosons}.

\begin{table}[h!]
 \resizebox{0.8\textwidth}{!}{
\begin{tabular}{ |c  | c  | c|  }
\hline
Force &   Mediator & Mass  \\
\hline
strong &   gluons $\Pgluon \in\{\Pgluon_{1},...,\Pgluon_{8}\}$   & 0   \\
\hline
electromagnetic &  photon \Pphoton & 0   \\
\hline
 \multirow{2}{*}{weak} &  W-bosons \PWpm & 80.4 \GeV   \\
                       &  Z-boson \PZ & 91.2 \GeV   \\
\hline
  \end{tabular}
}
 \caption[The three fundamental forces described by the SM. The mediator particle of each force is indicated along with its measured mass, where the uncertainties have been omitted\quad\cite{Thomson:2013zua,PDGBooklet}.]{The three fundamental forces described by the SM. The mediator particle of each force is indicated along with its measured mass, where the uncertainties have been omitted~\cite{Thomson:2013zua,PDGBooklet}.}
\label{tab:th:bosons}
  \end{table}

The \Pphoton and \Pgluon are massless, in contrast to the \PWpm and \PZ, which are massive. This difference is explained by the process of \emph{electroweak symmetry breaking} via the Brout-Englert-Higgs mechanism~\cite{Englert:1964et,Higgs:1964ia,Higgs:1964pj,Guralnik:1964eu,Higgs:1966ev,Kibble:1967sv} described in \Sec~\ref{sec:th:ewsb}. This mechanism introduces an additional scalar field, which implies the existence of a massive spin-$0$ particle, the Higgs boson.

\subsection{Gauge groups of the SM Lagrangian}
\label{sec:th:gauge}

The \SM is a \QFT, in particular a renormalisable gauge theory. 
The Lagrangian $\mathcal{L}_{\text{QFT}}$ of a \QFT describes the dynamics and interactions of its particles. %The equations of motion can be extracted using the \ELE. $\mathcal{L}_{\text{QFT}}$ 
It is constructed by considering the nature of the particles involved in the \QFT and imposing the symmetries of the theory.
N\"other's Theorem~\cite{Noether} states that for every symmetry in a Lagrangian, there is an associated conservation law. 
%For example, if a Lagrangian is invariant in time or space, this directly implies conservation of energy or momentum respectively within the theory. 
For example, if a theory respects conservation of energy or momentum, its Lagrangian must be invariant in time or space respectively. Imposing a symmetry in a Lagrangian places requirements on how the particles in the theory are allowed to propagate and interact. 

A gauge theory is a particular type of \QFT where local gauge transformations are a symmetry of the Lagrangian, leading to conservation of charge. Such gauge symmetries are of principal importance in particle physics, as they lead to the introduction of gauge fields, which generate the mediator particles. For this reason, the mediators of the forces are sometimes referred to as \emph{gauge bosons}. 
%The \SM is in fact a collection of gauge theories. 
Typically, a gauge transformation takes the form of shifting the phase of all wavefunctions. It is reasonable to require such transformations to leave the dynamics of the theory intact, since the phase of a wavefunction is never manifest in any physical observable. Thus the Lagrangian of any realistic theory should be \emph{gauge invariant}.
% i.e. symmetric under phase transformation operations.  

\subsubsection{Quantum Electrodynamics}
\label{sec:th:qed}
A simple example of a gauge theory is \QED. 
%A single fermion is considered here (others can be trivially added). 
This theory must incorporate fermions, which are described by the Dirac Lagrangian~\cite{griffiths2008introduction}:

\begin{equation}
\label{eq:th:dirac}
\mathcal{L}_{\textrm{fermion}} = i\overline{\psi} \gamma^{\alpha} \partial_{\alpha} \psi - m\overline{\psi}\psi,
\end{equation}

where $\psi$ is a Dirac spinor and $\overline{\psi}$ is its adjoint, $m$ is the mass of the fermion, $\gamma^{\alpha}$ represents the four Dirac gamma matrices and $\partial_{\alpha}$ is the 4-gradient. 
%Requiring \emph{global} gauge invariance means that we require that the Lagrangian is invariant under a global phase transformation
%\begin{equation}
%\label{eq:th:local_gauge_transform}
%\psi \rightarrow \psi'= \psi e^{i\theta},
%\end{equation}
%where $\theta$ is constant in time and space. This requirement is trivially satisfied by $\mathcal{L}_{\textrm{fermion}}$ :
%$$
%\mathcal{L}_{\textrm{fermion}} \rightarrow \mathcal{L}_{\textrm{fermion}}'=i\overline{\psi e^{i\theta}} \gamma^{\alpha} \partial_{\alpha} \psi e^{i\theta} - m\overline{\psi e^{i\theta}}\psi e^{i\theta}  =i \overline{\psi } e^{-i\theta} e^{i\theta}\gamma^{\alpha} \partial_{\alpha} \psi - m\overline{\psi} e^{-i\theta} e^{i\theta}\psi  = \mathcal{L}_{\textrm{fermion}},
%$$
%where it was possible to move $e^{i\theta}$ from the left to the right of the 4-gradient because $\theta$ is a constant. 

Requiring \emph{local gauge invariance} means that the Lagrangian should be invariant under transformations such as
$\psi \rightarrow \psi'= \psi e^{i\theta(x^{\mu})}$, where $\theta(x^{\mu})$ is an arbitrary differentiable function of space-time $x^{\mu}$. Applying the transformation to \Eq~\ref{eq:th:dirac} gives:
\begin{equation}
\label{eq:th:dirac_lagrangian_not_invariant_local_gauge_transf}
\mathcal{L}_{\textrm{fermion}} \rightarrow \mathcal{L}_{\textrm{fermion}}'=  \mathcal{L}_{\textrm{fermion}} - \overline{\psi} \gamma^{\alpha} \psi (\partial_{\alpha} \theta(x^{\mu})).
\end{equation}

Evidently $\mathcal{L}_{\textrm{fermion}}$ is not gauge invariant.
%even though this is a reasonable requirement for any theory which describes our universe. 
This is remedied by introducing an additional field $A_\alpha$. An extra term, $- g_{\textrm{EM}}\overline{\psi}\gamma^{\alpha}\psi A_{\alpha}$, is added to the Lagrangian to account for the interaction of the fermion with $A_\alpha$,
%$$
%\mathcal{L}_{\textrm{fermion+interaction}} = i\overline{\psi} \gamma^{\alpha} \partial_{\alpha} \psi - m\overline{\psi}\psi  - g_{\textrm{EM}}\overline{\psi}\gamma^{\alpha}\psi A_{\alpha},
%$$
where $g_{\textrm{EM}}$ is the strength of the interaction. Local gauge invariance is restored so long as $A_\alpha$ changes in the following way~\cite{griffiths2008introduction}: 

\begin{equation}
\label{eq:th:photon_gauge}
A_{\alpha} \rightarrow  A_{\alpha}' =  A_{\alpha} - \frac{1}{g_{\textrm{EM}}} \partial_{\alpha} \theta(x^{\mu}),
\end{equation}

%According to \Eq~\ref{eq:th:photon_gauge}, the interaction term transforms as:
%$$
%\mathcal{L}_{\textrm{interaction}} \rightarrow  \mathcal{L}_{\textrm{interaction}}' =  \mathcal{L}_{\textrm{interaction}}  + \overline{\psi} \gamma^{\alpha} \psi (\partial_{\alpha} \theta(\vec{x},t)),
%$$
%When applying the gauge transformation, the potential produces a $ + \overline{\psi} \gamma^{\alpha} \psi (\partial_{\alpha} \theta(\vec{x},t)) $ which exactly cancels out the additional term in \Eq~\ref{eq:th:dirac_lagrangian_not_invariant_local_gauge_transf}, thereby restoring gauge invariance.

The Lagrangian can also accommodate a term for $A_{\alpha}$ propagating freely through space. Since $A_{\alpha}$ is a 4-vector, it is described by the Proca equation for spin-1 bosons~\cite{griffiths2008introduction}:

\begin{equation}
\label{eq:th:proca}
\mathcal{L}_{\textrm{boson}} = -\frac{1}{16\pi} F^{\alpha\beta}F_{\alpha\beta} + \frac{1}{8\pi} m_{\textrm{boson}} A^{\alpha} A_{\alpha},
\end{equation}
where $F^{\alpha \beta} =(\partial^{\alpha} A^{\beta} - \partial^{\beta} A^{\alpha})$ and $m_{\textrm{boson}}$ is the mass of the spin-1 boson. 
\Eq~\ref{eq:th:proca} is locally gauge invariant so long as $m_{\textrm{boson}}=0$, i.e.~the boson is required to be massless.

%The full Lagrangian for QED is therefore composed of a Lagrangian for: a free fermion,  a free spin-1 boson and a term for interactions between the boson and the fermion. 

%\begin{equation}
%\label{eq:th:QED_lagrangian}
%\mathcal{L}_{\textrm{QED}} = \underbrace{i\overline{\psi} \gamma^{\alpha} \partial_{\alpha} \psi - m\overline{\psi}\psi}_{\textrm{free fermion}}   \overbrace{-\frac{1}{16\pi} F^{\alpha\beta}F_{\alpha\beta}}^{\textrm{free spin-1 boson}}  \underbrace{- g_{\textrm{EM}}\overline{\psi}\gamma^{\alpha}\psi A_{\alpha}}_{\textrm{interaction term}}.
%\end{equation}
It is convenient to define the \emph{covariant derivative}, incorporating the interaction term:

\begin{equation}
\label{eq:th:covariant_derivative}
D_{\alpha} = \partial_{\alpha} + i g_{\textrm{EM}}A_{\alpha}.
\end{equation}

The Lagrangian $\mathcal{L}_{\textrm{QED}}$ can then be written compactly as:

\begin{equation}
\label{eq:th:QED_lagrangian}
\mathcal{L}_{\textrm{QED}} = i\overline{\psi} \gamma^{\alpha} D_{\alpha} \psi - m\overline{\psi}\psi -\frac{1}{16\pi} F^{\alpha\beta}F_{\alpha\beta}.
\end{equation}

The full Lagrangian for \QED describes a free fermion, a free massless spin-1 boson and a completely determined term for interactions between the boson and the fermion. The boson is identified as the photon. The factor multiplying $g_{\textrm{EM}}$ is interpreted as the electric charge of the fermion.

The local gauge transformation is equivalent to applying a unitary $1\times1$ matrix to the wavefunction. The group of all such transformations is $U(1)$. It is a general result that the number of degrees of freedom in the underlying symmetry group dictates the number of additional bosons needed to keep the theory locally gauge invariant~\cite{griffiths2008introduction}. 

\subsubsection{Quantum Chromodynamics}

The strategy described in \Sec~\ref{sec:th:qed} can be used to derive the Lagrangian for \QCD, which codifies the dynamics of the strong force~\cite{griffiths2008introduction}.  %~\cite{Han:1965pf,Fritzsch:1973pi}. 
The situation is more complicated because the underlying group is not $U(1)$ but $SU(3)$, which has eight degrees of freedom. This leads to eight massless gluons. In \QED, the interaction between the boson and the fermion is dictated by electric charge; the analogue for \QCD is \emph{colour charge}. However, colour charge cannot be represented by a single quantity. Colours charges are instead linear combinations of three quantities, designated \emph{red, green} and \emph{blue}. Each \SM quark has three identical copies with different colour charge. Another important distinction is that $U(1)$ is abelian while $SU(3)$ is not. The consequence of this is that unlike the \QED photon, which carries no electric charge, the \QCD gluons do carry colour charge. Gluons can therefore feel the strong force and self-interact. This fact leads \QCD to display properties such as confinement of quarks and asymptotic freedom~\cite{PhysRevLett.30.1346,PhysRevLett.30.1343}. The gauge group for \QCD, $SU(3)_{\textrm{C}}$, is generally written with a subscript to indicate that it generates colour charge. 

\subsubsection{Electroweak unification}

%Although an equivalent of \QED exists for the weak force, \QFD, 
The development of a combined theory of the electromagnetic and weak forces, called \EWT, was a major achievement by Glashow, Weinberg and Salam~\cite{GlashowPartialSymmetries,WeinbergModelOfLeptons,SalamNobelSymposium}. %, thus superseding \QFD. 
\EWT considers the gauge group $SU(2)_{\textrm{L}} \times U(1)_{\textrm{Y}}$. The $SU(2)_{\textrm{L}}$ group has three degrees of freedom, so three gauge bosons are obtained from imposing local gauge invariance. The conserved charge is \emph{weak isospin}. This quantity is a vector, the third component of which is labelled $i_3$. The subscript in $SU(2)_{\textrm{L}}$ indicates that only left-handed particles carry nonzero weak isospin charge. The $U(1)_{\textrm{Y}}$ group behaves as in \QED, but generates \emph{weak hypercharge} $y$. The electric charge $q$ is related to weak isospin and weak hypercharge by the relation $q=y/2 + i_3$. The subscript in $U(1)_{\textrm{Y}}$ refers to the fact that hypercharge is the generated charge.

Right- and left-handed fermions are considered separately in \EWT. It is helpful to think of the right-handed fermion spinors as \emph{singlets}, or column vectors of one spinor. For example, the right-handed electron singlet is labelled $\Pe_{R}$. 
The left-handed fermions come in \emph{doublets} (a column vector of two fermion spinors), e.g.~:
$$
L_{\textrm{L}}=\binom{\Pnue}{\Pe}_{\textrm{L}},
$$
which is the left-handed lepton doublet containing the left-handed electron and electron neutrino. No right-handed neutrinos are included in this scheme. This is a deliberate feature which reflects the fact that no right-handed neutrinos have been observed or detected experimentally. However, the discovery of neutrino oscillations implies the existence of such neutrinos, so clearly the \SM is only an approximate theory. 

Imposing gauge invariance leads to the introduction of gauge fields: $W_{\alpha}^{1},W_{\alpha}^{2},W_{\alpha}^{3}$ and $B_{\alpha}$, for $SU(2)_{\textrm{L}}$ and $U(1)_{\textrm{Y}}$ respectively. The physical states observed in nature are mixtures of the underlying weak isospin and weak hypercharge gauge bosons:

\begin{equation}
\label{eq:th:electroweak_gauge_bosons}
\PWpm_{\alpha} = \sqrt{\frac{1}{2}}(W_{\alpha}^{1} \mp W_{\alpha}^{2}) ,\\
\PZ_{\alpha} = \cos \theta_{\textrm{W}} W_{\alpha}^{3} - \sin \theta_{\textrm{W}} B_{\alpha} ,\\
A_{\alpha} = \sin \theta_{\textrm{W}} W_{\alpha}^{3} + \cos \theta_{\textrm{W}} B_{\alpha} ,
\end{equation}

where $\theta_{\textrm{W}}$ is the Weinberg angle, which relates strengths of the electromagnetic ($g_{\textrm{EM}}$) and weak ($g_{\textrm{W}}$) forces via the relation $\tan \theta_{\textrm{W}} = g_{\textrm{W}} /g_{\textrm{EM}}$. The value of $\sin^2\theta_{\textrm{W}}$ has been determined experimentally to be $\sim 0.23$~\cite{PDGBooklet}.

%\EWT successfully presents the electromagnetic and weak forces as manifestations of the same underlying electroweak force generated by $SU(2)_{\textrm{L}} \times U(1)_{\textrm{Y}}$. 
 Quarks are also accommodated in this framework. For example, for the first generation of quarks, two right-handed quark singlets $u_{R}$ and $d_{R}$ and one left-handed quark doublet $Q_{L}$ (containing $u_{\textrm{L}}$ and $d_{\textrm{L}}$) are introduced. The full \SM, incorporating the electroweak and strong forces, is described by the gauge group $SU(3)_{\textrm{C}} \times SU(2)_{\textrm{L}} \times U(1)_{\textrm{Y}}$.
 
 
An issue arises when considering masses of particles. 
%In \QED where the mass of a fermion is given simply by the term $ -m\overline{\psi}\psi$ in the Lagrangian, but the equivalent term, for example for the electron, must account for both the right-handed and left-handed components, which transform differently since one is a singlet and the other is part of a doublet. This breaks the gauge invariance of the fermion mass terms.
The left- and right-handed components of the fermions transform as doublets and singlets respectively, so the usual fermion mass term is no longer gauge invariant.
Furthermore, the gauge bosons should be massless to preserve the gauge symmetry. This is the case for the photon and gluons, but the \PWpm and \PZ need to have masses to explain weak decays. These masses were later experimentally measured to be of the order of 90\GeV~\cite{PDGBooklet}. A mechanism is needed to account for the masses of these particles. 

\subsection{Electroweak Symmetry Breaking and the Higgs Mechanism}
\label{sec:th:ewsb}

The process which allows the \PWpm and \PZ to acquire a mass is \emph{electroweak symmetry breaking}. This occurs in the \SM via the Brout-Englert-Higgs mechanism~\cite{Englert:1964et,Higgs:1964ia,Higgs:1964pj,Guralnik:1964eu,Higgs:1966ev,Kibble:1967sv}. In this scheme, an additional complex scalar $SU(2)_{\mathrm{L}}$ doublet $\phi$ is introduced. The Lagrangian for $\phi$ is gauge invariant but the ground state is not: 
%Spin-0 boson singlets are described by solutions to the Klein-Gordon equation, which takes the following form in a Lagrangian:
%\begin{equation}
%\label{eq:th:klein_gordon_lagrangian}
%\mathcal{L}_{KG} = \frac{1}{2}(\partial_{\alpha} \Phi)^{*} (\partial^{\alpha} \Phi) - \frac{1}{2} (m_{\Phi})^{2} (\Phi^{*} \Phi)  ,
%\end{equation}
%where $m_{\Phi}$ is the mass of the sping-0 scalar boson $\Phi$.

\begin{equation}
\label{eq:th:higgs_lagrangian}
\begin{split}
\mathcal{L}_{\phi} = & \frac{1}{2}(D^{\textrm{EWK}}_{\alpha} \phi)^{*} (D^{\alpha}_{\textrm{EWK}} \phi) \text{ (kinetic term) }\\
 & + \mu^{2} (\phi^{*} \phi) - \frac{1}{4} \lambda^{2} (\phi^{*} \phi) ^{2}  \text{ (potential term) },
\end{split}
\end{equation}

where $\mu$ and $\lambda$ are constants. The covariant derivative $D^{{\textrm{EWK}}}_{\alpha}$, defined analogously to \Eq~\ref{eq:th:covariant_derivative}, has been used here to ensure gauge invariance. In this case, $D^{{\textrm{EWK}}}_{\alpha}$ accounts for the interactions of the electroweak gauge bosons $W^{1}_{\alpha}$, $W^{2}_{\alpha}$, $W^{3}_{\alpha}$ and $B_{\alpha}$:

\begin{equation}
\label{eq:th:full_covariant_derivative}
D^{{\textrm{EWK}}}_{\alpha} = \partial_{\alpha} -i g_{1} \frac{Y}{2} B_{\alpha} + i g_{2} \frac{\tau^{i}}{2} W^{i}_{\alpha},
\end{equation}
where $g_{\textrm{Y}}$ and $g_{\textrm{L}}$ refer to the coupling constants of the weak hypercharge and weak isospin fields respectively, $Y$ is the constant generator of the $U(1)_{\textrm{Y}}$ group and $\tau^{i}$ are the generators of the $SU(2)_{\textrm{L}}$ group. The coupling constants $g_{\textrm{Y}}$ and $g_{\textrm{L}}$ are related to the weak and electromagnetic coupling constants by $g_{\textrm{W}}=g_{\textrm{L}}$ and $g_{\textrm{EM}}= g_{\textrm{Y}} \cos \theta_{\textrm{W}}$. 

The first term of the Lagrangian corresponds to the kinetic part, while the second and third correspond to a potential. %Comparing to \Eq~\ref{eq:th:klein_gordon_lagrangian}, 
The term $ \frac{1}{2} \mu^{2} (\phi^{*} \phi)$ resembles a mass term, but it is not: the sign needs to be negative. If $\mu^{2}<0$, the potential has a non-zero \VEV, and the ground state is represented by a circle of minima. In order to rewrite the Lagrangian in terms of physical particles, an expansion around one of the minima is required. Since they are all equivalent, an arbitrary minimum is chosen:

\begin{equation}
\label{eq:th:higgs_vev}
\phi_{0} = \VEV = \frac{1}{\sqrt{2}} \binom{0}{v} ,
\end{equation}
where $v=\sqrt{- \mu^{2} / \lambda}$. At this stage, any of the minima lying upon the circle of ground states could have been chosen, but a particular choice had to be made. This step breaks the manifest symmetry in the physical states while preserving it in the Lagrangian. 

To obtain the physical states from this Lagrangian, a small perturbation field $H$ around the \VEV, is introduced. Any perturbation to the first component of the \VEV would represent a move to an equivalent minimum. Therefore such a perturbation can be ignored, and $H$ only acts on the second component:

\begin{equation}
\label{eq:th:higgs_vev_perturbation}
\phi = \frac{1}{\sqrt{2}} \binom{0}{v+H} ,
\end{equation}
which can be substituted back into \Eq~\ref{eq:th:higgs_lagrangian}. Expanding out the interactions with the gauge bosons from the covariant derivative, and expressing them in terms of the physical bosons using \Eq~\ref{eq:th:electroweak_gauge_bosons}, the equation becomes:

\begin{equation}
\begin{split}
%\mathcal{L}_{\phi} = \frac{1}{2} (\partial_{\alpha} H) (\partial^{\alpha} H) - \frac{1}{2} \mu^{2} H^2 + \frac{v^2}{8}( g_{\textrm{W}} \PWp_{\alpha}\PWp^{\alpha} + g_{\textrm{W}} \PWm_{\alpha}\PWm^{\alpha} + (g^2_{\textrm{W}}+g^2_{\textrm{EM}})  \PZ_{\alpha}\PZ^{\alpha}) + ...,
  \mathcal{L}_{\phi}  =  \frac{1}{2} & (\partial_{\alpha} H) (\partial^{\alpha} H) -  \mu^{2} H^2 \\
&  + \frac{v^2}{8}( g_{\textrm{W}} \PWp_{\alpha}\PWp^{\alpha} + g_{\textrm{W}} \PWm_{\alpha}\PWm^{\alpha} + (g^2_{\textrm{W}}+g^2_{\textrm{EM}}) \PZ_{\alpha}\PZ^{\alpha}) + ...
\end{split},
\end{equation}
\label{eq:th:higgs_lagrangian}
where terms which are not mass-like have been omitted.
%Now that the Lagrangian has ben re-written after expansion around a valid minimum, the physical states can be identified.
%it with the Klein-Gordon equation% in \Eq~\ref{eq:th:klein_gordon_lagrangian}. 
The first two terms can be identified as the Klein-Gordon equation for a massive scalar boson of mass $\sqrt{2}\mu$: this is the Higgs boson. The other terms represent the great success of the Brout-Englert-Higgs mechanism: the \PWpm and \PZ bosons have acquired masses. The absence of equivalent terms for the photon means that it does not acquire mass.

%Adding $\mathcal{L}_{\phi}$ to the \SM Lagrangians for \EWT and \QCD maintains local gauge invariance, but allows the weak force mediators to acquire mass. 
It is possible to add additional gauge invariant terms involving the scalar field. These are known as the Yukawa interactions. For example, in the case of the first generation leptons:

\begin{equation}
\label{eq:th:yukawa_coupling}
\begin{split}
  \mathcal{L}_{\textrm{Yukawa}} &= \kappa_{e} (\overline{L} \phi  e_{R} + \overline{e_{R}} \phi^{\dagger} L )\\
                       &=  \kappa_{e} v (\overline{e_{L}} e_{R} + \overline{e_{R}} e_{L} ) +  \kappa_{e}   (\overline{e_{L}} H e_{R} + \overline{e_{R}} H e_{L} ),
\end{split}
\end{equation}
where $\kappa_{e}$ is a real constant. The first term gives a mass to the electron and the second represents the interaction between the electron and the Higgs boson. In this way, the strength of the interaction is directly proportional to the mass of the particle which is considered. The neutrino does not acquire any mass via this mechanism. The scheme described above does not \emph{prescribe} the masses of the Higgs boson or fermions: these are free parameters and must be specified from experimental measurements.

To summarise, the Brout-Englert-Higgs mechanism adds gauge-invariant terms to the \SM Lagrangian which permit the $\PWpm$ and $\PZ$ bosons and the fermions to acquire masses while leaving the photon massless. The outcome is one additional spin-0 particle, the Higgs boson, the mass of which is not directly predicted by the theory.

 
\section{Higgs boson phenomenology}
\subsection{History of Higgs boson searches}
\label{sec:th:higgs_searches}

Since the Higgs boson was postulated in the 1960s, there have been many efforts to try to detect it. Theoretical considerations precluded large Higgs boson masses above the order of 1\TeV~\cite{Heller:1993yv}. 
%Experiments conducted before the start of \LEP operation excluded a Higgs boson of mass below the order of 10\GeV~\cite{Wu:2014vva}. 
A Higgs boson of mass below about 10\GeV was already excluded before the start of \LEP~\cite{Wu:2014vva}.
Direct searches at \LEP and the Tevatron excluded a Higgs boson mass below 114\GeV~\cite{Barate:2003sz,TEVNPH:2012ab}, and precision measurements of the electroweak parameters suggested that the Higgs boson mass should be below 200\GeV~\cite{Renton:2004wd}. The search for the Higgs boson was one of the goals prompting the construction of the \LHC at \CERN. Two multi-purpose detectors, \ATLAS and \CMS, were designed with the Higgs observation as one of their main physics goals. In 2012, the two experiments jointly announced the observation of a Higgs-like particle, ending a 50-year interval between postulation and discovery~\cite{CMSHDisc,ATLASHDisc}. The most precise measurement of the Higgs boson's mass to date is $\mH=125.09\pm0.24\GeV$~\cite{PhysRevLett.114.191803}.

\subsection{Higgs boson production at the LHC}
\label{sec:th:higgs_production_modes}

According to the \SM, the Higgs boson interacts with particles proportional to their masses. Four types of process can lead to the production of a Higgs boson in \pp collisions at the \LHC. The Feynman diagrams for these processes can be seen in \Fig~\ref{fig:theory:higgsproduction}. The most likely production mode for $\mH=125\GeV$ is \ggH via a loop of top quarks, which has a \crosssection of approximately 49\pb at 13\TeV. The other production modes are: \VBF at 3.8\pb; \VH at 2.3\pb; and \ttH at 0.5\pb~\cite{LHCHXSWGYR4}. The \VH production process can be further split into \WH (1.4\pb) and \ZH (0.6\pb), which have different final states.

  \begin{figure}[h!]

  \centering
  \subfloat[]{
    \begin{fmfgraph*}(150,100)
      \fmfright{o0,o1,ox,o3,o4}
      \fmfleft{i0,i1,ix,i3,i4}
      \fmf{phantom,tension=4/3}{i1,v1,o1}
      \fmf{phantom,tension=4/3}{i3,v2,o3}
      \fmffreeze
      \fmf{gluon,tension=4/3}{i1,v1}
      \fmf{gluon,tension=4/3}{i3,v2}
      \fmf{fermion,tension=0,label=$t$,label.side=left}{v1,v2}
      \fmf{fermion,tension=2/3,label=$t$,label.side=left}{v2,v3}
      \fmf{fermion,tension=2/3,label=$t$,label.side=left}{v3,v1}
      \fmf{dashes}{v3,ox}
      \fmflabel{$g$}{i1}
      \fmflabel{$g$}{i3}
      \fmflabel{$H$}{ox}
  \end{fmfgraph*} 
  } \hspace{1cm}
  \subfloat[]{
    \begin{fmfgraph*}(150,100)
      \fmfright{o0,o1,o2,o3,o4}
      \fmfleft{i0,i1,i2,i3,i4}
      \fmf{phantom,tension=4/3}{i0,v1}
      \fmf{phantom,tension=2/3}{v1,o0}
      \fmf{phantom}{v1,v3}
      \fmf{phantom}{v2,v3}
      \fmf{phantom}{v3,o2}
      \fmf{phantom,tension=4/3}{i4,v2}
      \fmf{phantom,tension=2/3}{v2,o4}
      \fmffreeze
      \fmf{fermion,label=$q$,label.side=right}{i0,v1,o0}
      \fmf{fermion,label=$q'$,label.side=left}{i4,v2,o4}
      %\fmf{fermion,label=$q$}{v1,o1}
      %\fmf{fermion,label=$q$}{v2,o2}
      \fmf{boson,label=$Z/W$,label.side=right}{v1,v3}
      \fmf{boson,label=$Z/W$,label.side=left}{v2,v3}
      \fmf{dashes}{v3,o2}
      \fmflabel{$H$}{o2}
  \end{fmfgraph*} 
  }\\\vspace{1cm}
  \subfloat[]{
    \begin{fmfgraph*}(150,100)
      \fmfright{o0,o1,o2,o3,o4}
      \fmfleft{i0,i1,i2,i3,i4}
      \fmf{phantom}{i0,v1,i4}
      \fmf{phantom}{v1,v2}
      \fmf{phantom}{o0,v2,o4}
      \fmffreeze
      \fmf{fermion,label=$q$,label.side=right}{i0,v1,i4}
      \fmf{boson,label=$Z/W$,label.side=right}{v1,v2}
      \fmf{boson,label=$Z/W$,label.side=left}{v2,o0}
      \fmf{dashes,label=$H$,label.side=left}{v2,o4}
      %\fmflabel{$H$}{o2}
  \end{fmfgraph*}
  } \hspace{1cm}
  \subfloat[]{
    \begin{fmfgraph*}(150,100)
      \fmfright{o0,o1,o2,o3,o4}
      \fmfleft{i0,i1,i2,i3,i4}
      \fmf{phantom,tension=4/3}{i0,v1}
      \fmf{phantom,tension=2/3}{v1,o0}
      \fmf{phantom}{v1,v3}
      \fmf{phantom}{v2,v3}
      \fmf{phantom}{v3,o2}
      \fmf{phantom,tension=4/3}{i4,v2}
      \fmf{phantom,tension=2/3}{v2,o4}
      \fmffreeze
      \fmf{gluon,label=$\quad g$,label.side=right}{i0,v1}
      \fmf{gluon,label=$g$,label.side=left}{i4,v2}
      \fmf{fermion,label=$t$,label.side=right}{o0,v1}
      \fmf{fermion,label=$t$,label.side=right}{v1,v3}
      \fmf{fermion,label=$t$,label.side=right}{v3,v2}
      \fmf{fermion,label=$t$,label.side=left}{v2,o4}
      %\fmf{fermion,label=$q$}{v1,o1}
      %\fmf{fermion,label=$q$}{v2,o2}
      \fmf{dashes}{v3,o2}
      \fmflabel{$H$}{o2}
  \end{fmfgraph*} 
  }\vspace{1cm}
  \caption{Higgs production modes at the LHC: (a) gluon-gluon fusion, via a loop of top quarks, (b) vector boson fusion, with associated quark production, (c) associated vector boson production with either the \PZ or \PW boson and (d) top quark fusion with associated top quark production. }
  \label{fig:theory:higgsproduction}
  \end{figure}

\subsection{Higgs boson decays}
\label{sec:th:higgs_decays}

The \SM Higgs boson can decay either directly to pairs of particles, or via virtual loops.
In direct decays to pairs of particles, the  branching ratios are proportional to the mass of the decay product for fermions and the square root of the mass of the decay product for vector boson. The most likely direct decay modes to massive particles for a \SM Higgs boson with $\mH=125\GeV$ are $\PH \rightarrow \Pbottom \Pbottom$ (58.2\%); $\PH \rightarrow \PW \PW^{*}$ (21.4\%, where $\PW^{*}$ refers to a virtual $\PW$); $\PH \rightarrow \Ptau \Ptau$ (6.3\%); $\PH \rightarrow \Pcharm \Pcharm$ (2.8\%); $\PH \rightarrow \PZ \PZ^{*}$ (2.6\%, where $\PZ^{*}$ refers to a virtual $\PZ$)~\cite{LHCHXSWGYR4}. The production of a pair of $t$ quarks is strongly suppressed by kinematics. The Higgs boson's couplings to electrons, muons, up quarks, down quarks and strange quarks are very small due to the mass of the decay products. In the \SM, the neutrinos do not have mass, so no coupling to the Higgs boson is predicted.
In addition, Higgs boson decays can occur via a loop of virtual massive particles to a pair of gluons (8.2\%), to a pair of photons (0.23\%) or to $\PZ\gamma$ (0.02\%)~\cite{LHCHXSWGYR4}. 

%During the search for the Higgs boson in the first run of the \LHC, the five decay channels which were of most interest were the decays to  
%Despite the low branching fraction, the decay \Hgg played a key role in the discovery of the Higgs boson. 
For the \CMS and \ATLAS detectors, two channels are particularly suited to studies of the Higgs boson despite their relatively low branching fractions. These are \Hgg and $\PH \rightarrow \PZ \PZ^{*} \rightarrow 4 \Plepton $ (where $\Plepton$ refers to leptons, and the rate is reduced because the branching fraction of $\PZ\rightarrow \Plepton\Plepton$ needs to be taken into account). The fact that both detectors are able to reconstruct the energy an transverse momenta of electrons, muons and photons mean that a narrow Higgs boson mass peak can be reconstructed in these two channels. The other channels which contributed, to a lesser extent, to the discovery of the Higgs boson were $\PH \rightarrow \Pbottom \Pbottom$, $\PH \rightarrow \PW \PW^{*} $ and  $\PH \rightarrow \Ptau \Ptau$. %This is because it has a very clean signature of two highly energetic photons in the detector, with an irreducible but controllable \SM background. This is in contrast with some of the other more frequent decay modes, where 
These other channels have a higher rate but difficulties in reconstructing the decay products or excessive noise from the \LHC \pp collisions drastically reduce the experimental sensitivity. 



\subsection{Studying the Higgs boson using the \Hgg decay}
%%Need to find a witty title here

The work presented in this thesis focusses on the observation of the Higgs boson and a measurement of its properties via the \Hgg decay, for which the leading order Feynman diagrams are shown in \Fig~\ref{fig:theory:higgstogammagamma}. The data collected by the \CMS detector are selected for such measurements if they contain two photons which are candidates to have originated from a Higgs boson decay. The analysis of these data involves defining categories which target the individual production modes of the Higgs boson presented in \Sec~\ref{sec:th:higgs_production_modes}. Indeed, the final states of each production mode lead to different signatures in the detector, which can be exploited for the purposes of categorisation. This approach not only increases the overall sensitivity of the analysis, but also gives a handle with which to probe the SM predictions for the rate at which the Higgs boson is produced in each mode, as well as the strength of its interaction with SM particles. Many extensions to the \SM predict variations in these properties, so such measurements could put limits on new models or could yield clues to the nature of physics beyond the \SM.

\begin{figure}[h!]
\subfloat[]{
    \begin{fmfgraph*}(100,100)
      \fmfright{o0,o2,ox,o3,o5}
      \fmfleft{i0,i3,i1,i4,i6}
      %\fmf{phantom,tension=4/3}{o2,v1,i3}
      %\fmf{phantom,tension=4/3}{o3,v2,i4}
      \fmf{phantom,tension=2}{o2,v1}
      \fmf{phantom,tension=1}{v1,i3}
      \fmf{phantom,tension=2}{o3,v2}
      \fmf{phantom,tension=1}{v2,i4}
      \fmffreeze
      \fmf{photon,tension=4/3}{o2,v1}
      \fmf{photon,tension=4/3}{o3,v2}
      \fmf{fermion,tension=0,label=$t$}{v1,v2}
      \fmf{fermion,tension=2/3,label=$t$}{v2,v3}
      \fmf{fermion,tension=2/3,label=$t$,label.side=right}{v3,v1}
      \fmf{dashes}{v3,i1}
      \fmflabel{$\gamma$}{o2}
      \fmflabel{$\gamma$}{o3}
      \fmflabel{$H$}{i1}
  \end{fmfgraph*} 
  }\\
\hspace{1cm}\\
\subfloat[]{
    \begin{fmfgraph*}(100,100)
      \fmfright{o0,o2,ox,o3,o5}
      \fmfleft{i0,i3,i1,i4,i6}
      \fmf{phantom,tension=2}{o2,v1}
      \fmf{phantom,tension=1}{v1,i3}
      \fmf{phantom,tension=2}{o3,v2}
      \fmf{phantom,tension=1}{v2,i4}
      \fmffreeze
      \fmf{photon,tension=1}{o2,v1}
      \fmf{photon,tension=1}{o3,v2}
      \fmf{boson,tension=0,label=$W$}{v1,v2}
      \fmf{boson,tension=1,label=$W$}{v2,v3}
      \fmf{boson,tension=1,label=$W$,label.side=right}{v3,v1}
      \fmf{dashes}{v3,i1}
      \fmflabel{$\gamma$}{o2}
      \fmflabel{$\gamma$}{o3}
      \fmflabel{$H$}{i1}
  \end{fmfgraph*} 
  }\\
\hspace{1cm}\\
\subfloat[]{
    \begin{fmfgraph*}(100,100)

      \fmfleft{i} \fmfright{of1,o1,of,o2,of2}
    \fmf{dashes,tension=2}{i,v1} 
    \fmf{boson}{v2,o1}
    \fmf{boson}{v2,o2}
    \fmf{boson,label=$W$,left=0.8}{v1,v2,v1}
      \fmflabel{$\gamma$}{o1}
      \fmflabel{$\gamma$}{o2}
      \fmflabel{$H$}{i}

    \end{fmfgraph*} 
}
\caption{A Higgs boson decaying to photons via a loop of top quarks (a) or via loops of \PW bosons (b, c).}
\label{fig:theory:higgstogammagamma}
\end{figure}



