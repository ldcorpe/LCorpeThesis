%\begin{colophon}
%  This thesis was made in \LaTeXe{} using the ``hepthesis'' class~\cite{hepthesis}.
%\end{colophon}

%UPDATE THIS
%% You're recommended to use the eprint-aware biblio styles which
%% can be obtained from e.g. www.arxiv.org. The file thesis.bib
%% is derived from the source using the SPIRES Bibtex service.
\bibliographystyle{lucas_unsrt}
\bibliography{thesis}
\newpage

%%%%% \renewcommand{\listfigurename}{List of figures}
%%%%% \renewcommand{\listtablename}{List of tables}

%%%%% \listoffigures
%%%%% \listoftables

\chapter*{List of Acronyms}
\addcontentsline{toc}{chapter}{Acronyms}
\markboth{ACRONYMS}{}

\begin{acronym}
\vspace{0.5cm}

%%% Louie Corpe User defined
%\acro{pp}{(proton-proton)}
%%\acro{\PbPb}{(lead-lead)}


%theory
\acro{VBF}{vector boson fusion}
\acro{ggH}{gluon-gluon fusion}
\acro{VH}{vector boson associated production}
\acro{ttH}{top quark fusion and associated production}
\acro{ZH}{\PZ boson associated production}
\acro{WH}{\PW boson associated production}
\acro{DM}{dark matter}
\acro{CL}{confidence level}
\acro{ISR}{initial state radiation}
\acro{EFT}{effective field theory}
\acro{VEV}{vacuum expectation value}

%detector
\acro{CERN}{the European Organization for Nuclear Research}
\acro{CoM}{centre-of-mass}
\acro{LHC}{the Large Hadron Collider}
\acro{CMS}{Compact Muon Solenoid}
\acro{ATLAS}{A Toroidal LHC Apparatus}
\acro{ALICE}{A Large Ion Collider Experiment}
\acro{LHCb}{Large Hadron Collider beauty}
\acro{LEP}{the Large Electron-Positron Collider}
\acro{PSB}{Proton Synchrotron Booster}
\acro{PS}{Proton Synchrotron}
\acro{SPS}{Super Proton Synchrotron}
\acro{LINAC2}{Linear Accelerator 2}
\acro{PU}{pile-up}
\acro{SM}{standard model}
\acro{BSM}{beyond the SM}
\acro{QFT}{quantum field theory}
\acro{EWT}{electroweak theory}
\acro{QED}{quantum electrodynamics}
\acro{QFD}{quantum flavourdynamics}
\acro{QCD}{quantum chromodynamics}
\acro{ELE}{Euler-Lagrange equation(s)}
\acro{ECAL}{electromagnetic calorimeter}
\acro{ES}{preshower}
\acro{HCAL}{hadron calorimeter}
\acro{EB}{ECAL barrel}
\acro{EE}{ECAL endcaps}
\acro{HB}{hadron calorimeter barrel}
\acro{HE}{hadron calorimeter endcaps}
\acro{HF}{forward hadron calorimeter}
\acro{HO}{outer hadron calorimeter}
\acro{L1T}{level-1 trigger}
\acro{HLT}{high-level trigger}
\acro{CSCs}{cathode strip chambers}
\acro{DTs}{drift tubes}
\acro{RPCs}{resistive plate chambers}
\acro{WLCG}{Worldwide LHC Computing Grid}
\acro{ADC}{analogue to digital converters}
\acro{APDs}{avalanche photon-diodes}
\acro{VPTs}{vacuum photon-triodes}


%obj
\acro{SC}{supercluster}
\acro{PV}{primary vertex}
\acro{CTF}{combinatorial track finder}
\acro{DA}{``deterministic annealing''}
\acro{PF}{Particle flow}
\acro{GSF}{Gaussian sum filter}
\acro{BDT}{boosted decision tree}
\acro{MC}{Monte Carlo}
\acro{HPS}{hadron plus strips}
\acro{JES}{jet energy scale}


\acro{EM}{electromagnetic}


%prompt
\acro{CJV}{central jet veto}
\acro{MVA}{multi-variate analysis}
\acro{JER}{jet energy resolution}
\acro{UES}{unclustered energy scale}


%parked
\acro{CSV}{combined secondary vertex}


\acro{4FS}{four-flavour scheme}
\acro{5FS}{five-flavour scheme}
\acro{DAQ}{data acquisition}
\acro{LHCHXSWG}{LHC Higgs Cross Section Working Group}
\acro{LO}{leading order}
\acro{MPF}{missing transverse energy projection fraction}
\acro{MPI}{multi-parton interaction}
\acro{MSSM}{minimal supersymmetric standard model}
\acro{NLO}{next-to-leading order}
\acro{NNLO}{next-to-next-to-leading order}
\acro{NNLL}{next-to-next-to-leading logarithmic}
%\acro{pdf}{probability density function}
\acro{PDF}{parton distribution function}
\acro{RF}{radio frequency}
\acro{SSV}{simple secondary vertex}
\acro{TEC}{tracker endcaps}
\acro{TIB}{tracker inner barrel}
\acro{TID}{tracker inner disks}
\acro{TOB}{tracker outer barrel}
\acro{UE}{underlying event}
\end{acronym}

%% I prefer to put these tables here rather than making the
%% front matter seemingly interminable. No-one cares, anyway!

%% If you have time and interest to generate a (decent) index,
%% then you've clearly spent more time on the write-up than the 
%% research ;-)
%\printindex
