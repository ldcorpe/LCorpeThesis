\chapter{Introduction}
\label{chap:intro}

The discovery of the Higgs boson in 2012~\cite{CMSHDisc,ATLASHDisc} cemented the \SM of particle physics as one of the most successful theories of modern science in terms of its predictive power and agreement with experimental observations. The \SM describes the universe in terms of matter particles and fundamental forces, which are mediated by carriers. The great success of the \SM was to place all forces (except gravity) into one framework, and to use the Higgs mechanism to explain the manifest breaking of symmetry between the electromagnetic and weak forces. An overview of the theory and its predictions for the properties of the Higgs boson are given in \Chap~\ref{chap:theory}. In particular, this chapter details some of the reasons why the\Hgg decay is an excellent channel with which to study the Higgs boson. %, and look for any deviations from its expected behaviour.
%Despite its successes, however, the \SM is ostensibly incomplete or approximate. It does not leave room for neutrino masses or the existence of dark matter, and also does not account for the gravitational force or the matter-antimatter asymmetry in the universe. In short, there must exist new physics beyond the SM. Many such models have been proposed. In some theories, the Higgs boson is a composite particle or part of a wider family of Higgs bosons. Other theories suggest that it could be a 'portal' to new physics as it could interact with unknown particles with which regular matter cannot. All such theories would have measurable effects on the Higgs boson's production rate, interactions with other particles or even the number of Higgs bosons. 

The \LHC, as well as the two multi-purpose detectors \CMS and \ATLAS, were designed with the discovery of the Higgs boson as one of their primary objectives. In the case of \CMS, a key feature of the design is the \ECAL, which was conceived with the study of \Hgg in mind. The \LHC and the \CMS detector are described in \Chap~\ref{chap:detector}. In 2014, \CMS produced a standalone observation of the Higgs boson decaying to photons, using $24.8\ifb$ of data collected at $7$ and $8\TeV$~\cite{LegacyHgg}. In 2013, the \LHC began a two-year shutdown period, during which key upgrades to the accelerator and detectors were implemented. The hadron collider has now started up again, colliding particles at $13\TeV$. 
\ifNewAnalysis
The dataset analysed in this thesis contains \thisanalysislumi\ifb collected in this regime. This dataset corresponds to approximately three times more data than the one used to produce the previous \Hgg study~\cite{CMS-PAS-HIG-16-020}.
\else
The dataset analysed in this thesis contains \thisanalysislumi\ifb collected in this regime. The rates of the Higgs boson production processes are expected to evolve with the collision energy. For this reason, it so happens that the analysed dataset has a similar statistical sensitivity to the entire $7$ and $8\TeV$ dataset.
An important first task for the \CMS collaboration is therefore to confirm the existence of the Higgs boson in the $13\TeV$ data, and to check if the Higgs boson's properties evolve as expected.  
\fi

The analysis described in this thesis follows closely the methods used in previous studies of the \Hgg decay at \CMS~\cite{LegacyHgg,CMS-PAS-HIG-15-005,CMS-PAS-HIG-16-020}. This involves the selection and reconstruction of data containing two photons which are \Hgg candidates. This procedure is detailed in \Chap~\ref{chap:reconstruction}. The selected data are then categorised to optimise the overall sensitivity of the analysis on one hand, and to target specific Higgs boson production processes on the other hand. This scheme is described in \Chap~\ref{chap:categorisation}. 

The categorised data and simulation samples are used to produce models of how the \Hgg signal and background are expected to be manifested in the \mgg spectrum. This task, which includes modelling the sources of systematic uncertainty which enter the analysis, were the specific area of responsibility of the author. \Chap~\ref{chap:model} describes the procedure by which these models are produced. It also contains a description of two new techniques which were developed by the author.

Finally, the results of the analysis are detailed in \Chap~\ref{chap:statandresults}. They entail a new observation of the Higgs boson, and confirm that the measured properties of the new particle agree with the \SM expectation for $13\TeV$ within uncertainties 
\ifNewAnalysis
\else
, which are largely dominated by the statistical component.
\fi
In \Chap~\ref{chap:conclusions}, a discussion on the conclusions which can be drawn from these results is provided, alongside a view to the future of Higgs boson physics.


