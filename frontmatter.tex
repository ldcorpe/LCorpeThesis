%% Title
%??UPDATE THIS
\titlepage[Imperial College London\\Department of Physics]%
{A dissertation submitted to Imperial College London\\
for the degree of Doctor of Philosophy}
\newpage
The copyright of this thesis rests with the author and is made available under a Creative Commons
Attribution Non-Commercial No Derivatives licence. Researchers are free to copy, distribute or
transmit the thesis on the condition that they attribute it, that they do not use it for commercial
purposes and that they do not alter, transform or build upon it. For any reuse or redistribution,
researchers must make clear to others the licence terms of this work.

%% Abstract
\begin{abstract}%[\smaller \thetitle\\ \vspace*{1cm} \smaller {\theauthor}]
  %\thispagestyle{empty}
A study of \Hgg with \thisanalysislumi\ifb of data from \pp collisions at $\sqrt{s}=13\TeV$ is presented. The data were collected at the start of LHC \RunII by the CMS experiment between April and \finaldatatakingmonth 2016. The result is a new standalone observation of the Higgs boson in the diphoton decay channel. 
The significance of the excess is observed to be $\obsSigAtRunIBF\sigma$ ($\expSigAtRunIBF\sigma$ expected) at the combined \RunI best-fit Higgs boson mass of $\mH=125.09\GeV$. The maximum significance of $\obsSigAtMin\sigma$ ($\expSigAtMin\sigma$ expected) is observed at $\mH=\bestFitGlobalMH\GeV$.
The measured Higgs boson signal strength is $\obsMuBreakdown$. 
The signal strength is measured separately for: fermionic and bosonic production modes, giving $ \obsMuV$ and $\obsMuF$; 
\ifNewAnalysis
and for the ggH, VBF, ttH and VH production modes, giving $\obsMuggH$, $\obsMuVBF$, $\obsMuttH$, and $\obsMuVH$, .
\else
and for the ggH, VBF and ttH production modes, giving $\obsMuggH$, $\obsMuVBF$, and $\obsMuttH$.
\fi
Measurements of the Higgs boson coupling modifiers yield $\obskF$, $\obskV$ for the fermionic and bosonic coupling modifiers and $\obskGlu$, $\obskPho$ for the gluon and photon effective coupling modifiers.
All measurements are found to be consistent with the theoretical expectation for a SM-like Higgs boson.
\end{abstract}

%% Declaration
\begin{declaration}
The results presented in this thesis are the culmination of work which I certify to be my own, although it is based in part on studies by others. The theory described in \Chap~\ref{chap:theory} has been summarised in my words, but was produced by theorists. The LHC and CMS design and performance detailed in \Chap~\ref{chap:detector} are the result of studies by other experimentalists, although I was involved in the \RunI photon energy resolution measurements (see \Sec~\ref{sec:cms:ecal:overview}) and \RunII ECAL calibration (see \Sec~\ref{sec:cms:ecal:calibration}).
\mbox{\Chap\s}~\ref{chap:reconstruction} and~\ref{chap:categorisation} are the work of various individuals in the CMS collaboration and \Hgg group, of which I am a member. Some of the procedures are therefore the work of my collaborators.

The bulk of my personal work is presented in \Chap\s~\ref{chap:model} and~\ref{chap:statandresults}, which detail the signal and background modelling, handling of systematic uncertainties, statistical interpretation of the data and production of results. Indeed, these were my personal areas of responsibility in the CMS \Hgg group. 
\ifNewAnalysis
I was responsible for these topics for two studies of \Hgg with \RunII data~\cite{CMS-PAS-HIG-15-005,CMS-PAS-HIG-16-020}. The analysis presented here considers substantially more data. It also includes significant developments to the parametric signal modelling techniques which were conceived and implemented by me, and which I used to produce the results, but which do not feature in previous results. 
\else
I was responsible for these topics for two studies of \Hgg with \RunII data~\cite{CMS-PAS-HIG-15-005,CMS-PAS-HIG-16-020}, the latter being the basis for this thesis. These chapters also include significant developments to the parametric signal modelling techniques which were conceived and implemented by me, and which I used to produce the results, but which do not feature in~\cite{CMS-PAS-HIG-16-020}. 
\fi

Where ideas or figures from others are used, appropriate sources are referenced. Figures labelled ``\textbf{CMS}'' are taken from CMS publications or preliminary public documents, including those produced by me, and are referenced appropriately. The figures and results from the analysis I have performed in this thesis have been endorsed by the CMS collaboration. %\vspace*{0.5cm}
 \begin{flushright}
    Louie Dartmoor Corpe
  \end{flushright}
\end{declaration}


%% Acknowledgements
\begin{acknowledgements}
I would like to thank my parents, Lizzie and Andrew, who brought me up in a household where curiosity was encouraged. Dad, your bedtime stories about the Big Bang, entangled quarks (I imagined them as ducks flipping in unison) and black holes put me on this path. And Mum, I could not have come this far without your infectious enthusiasm, your drive to organise specialised tuition, your motivation to take part in science competitions and events, and importantly, your encouragement for me to have other hobbies! Thank you both for supporting me intellectually, emotionally (and financially!) throughout my education. I hope that despite your differences you can be united in sharing the pride of this achievement with me. Thanks also to my siblings, Olly and Margaux, whose friendship and humour have made the journey so enjoyable.

Thank you to my friends: those outside of academia, who have had to endure my endless ramblings about the glory of science; and those inside of academia, who have had to endure my endless ramblings about the dire state of politics. In particular, cheers to Adam, Adinda, Federico and Matthew for keeping me sane during our LTA. Although I'm still upset I never managed to win a game of Catan against you.

Thank you Paul and Chris, the best supervisors I could have hoped for: you provided fascinating projects to work on; always had time to give insightful feedback; and were \emph{very} generous with beer and coffee. On that topic, cheers to Seth, who was equally generous. Thanks for teaching me how to be a researcher, and for making Physics fun through the good times (\&) and the bad (holidayspirit). I'd like to express my gratitude to the IC HEP group and the CMS collaboration for providing an outstanding environment to do science in, and to the STFC for funding my studies and LTA. 

Finally, thank you to my fianc\'ee Emma, whose love and support have guided me through the highs and lows. Thank you for enduring the stress, the intervals of spatial separation and the ruined holidays. Conversely, thank you for enjoying the adventure and excitement, the new experiences and the skiing. Above all, thank you for being proud of my research and achievements, and for pushing me to follow my dreams. 

\end{acknowledgements}


%% Preface
%\begin{preface}
%\end{preface}

%% ToC
\tableofcontents

\renewcommand{\listfigurename}{List of figures}
\renewcommand{\listtablename}{List of tables}

\listoffigures
\listoftables

%% Strictly optional!
\frontquote%
{There are more things in heaven and earth, Horatio, \\
Than are dreamt of in your philosophy.} 
  {Hamlet (William Shakespeare)}
